% -*- mode: noweb; noweb-default-code-mode: R-mode; -*-
%\VignetteIndexEntry{Diversity analysis in vegan}
\documentclass[a4paper,10pt]{amsart}
\usepackage{ucs}
\usepackage[utf8x]{inputenc}
\usepackage[T1]{fontenc}
\usepackage{graphicx}
\usepackage{sidecap}
\setlength{\captionindent}{0pt}
\usepackage{url}


\title{Vegan: ecological diversity}
\author{Jari Oksanen}
\date{$ $Id: diversity-vegan.Rnw 205 2008-02-15 06:08:56Z jarioksa $ $
  processed with vegan 1.11-0
  in R version 2.6.2 (2008-02-08) on \today}
\usepackage{/usr/local/lib/R/share/texmf/Sweave}
\begin{document}
\setkeys{Gin}{width=0.55\linewidth}


\maketitle

\tableofcontents

\noindent The \texttt{vegan} packages has two major components:
multivariate analysis, mainly ordination, and methods for diversity
analysis of ecological communities.  This document gives an
introduction to the latter.  Ordination methods are covered in other
documents.  Many of the diversity functions were written by Roeland
Kindt and Bob O'Hara.

Most diversity methods assume that data are counts of individuals.
The methods are used with other data types, and some people argue that
biomass or cover are more adequate units than counts of individuals of
variable sizes.  However, this document only uses a data set with
counts: stem counts of trees on 1ha plots in the Barro Colorado
Island.  The following steps make these data available for the
document:
\begin{Schunk}
\begin{Sinput}
> library(vegan)
> data(BCI)
\end{Sinput}
\end{Schunk}

\section{Diversity indices}

Function \texttt{diversity} finds the most commonly used diversity
indices:
\begin{align}
H &= - \sum_{i=1}^S p_i \log_b  p_i & \text{Shannon--Weaver}\\
D_1 &= 1 - \sum_{i=1}^S p_i^2  &\text{Simpson}\\
D_2 &= \frac{1}{\sum_{i=1}^S p_i^2}  &\text{inverse Simpson}
\end{align}
where $p_i$ is the proportion of species $i$, and $S$ is the number of
species so that $\sum_{i=1}^S p_i = 1$, and $b$ is the base of the
logarithm.  It is most common to use natural logarithms (and then we
mark index as $H'$), but $b=2$ has
theoretical justification. Shannon index is calculated with:
\begin{Schunk}
\begin{Sinput}
> H <- diversity(BCI)
\end{Sinput}
\end{Schunk}
which finds diversity indices for all sites.

\texttt{Vegan} does not have indices for evenness (equitability), but
the most common of these, Pielou's evenness $J = H'/\log(S)$ is easily
found as:
\begin{Schunk}
\begin{Sinput}
> J <- H/log(specnumber(BCI))
\end{Sinput}
\end{Schunk}
where \texttt{specnumber} is a simple \texttt{vegan} function.

\texttt{Vegan} also can estimate Rényi diversities of order $a$:
\begin{equation}
H_a = \frac{1}{1-a} \log \sum_{i=1}^S p_i^a
\end{equation}
or the corresponding Hill numbers $N_a = \exp(H_a)$.  Many common
diversity indices are special cases of Hill numbers: $N_0 = S$, $N_1 =
\exp(H')$, $N_2 = D_2$, and $N_\infty = 1/(\max p_i)$.  We select a
random subset of five sites for Rényi diversities:
\begin{Schunk}
\begin{Sinput}
> k <- sample(nrow(BCI), 6)
> R <- renyi(BCI[k, ])
\end{Sinput}
\end{Schunk}
We can really regard a site more diverse if all of its Rényi
diversities are higher than in another site.  We can inspect this
graphically using the standard \texttt{plot} function for the
\texttt{renyi} result(Fig. \ref{fig:renyi}).
\begin{SCfigure}
\includegraphics{diversity-vegan-007}
\caption{Rényi diversities in six randomly selected plots. The plot
  uses Trellis graphics with a separate panel for each site. The dots
  show the values for sites, and the lines the extremes and median in
  the data set.}
\label{fig:renyi}
\end{SCfigure}

Finally, the $\alpha$ parameter of Fisher's log-series can be used as
a diversity index:
\begin{Schunk}
\begin{Sinput}
> alpha <- fisher.alpha(BCI)
\end{Sinput}
\end{Schunk}

\section{Rarefaction}

Species richness increases with sample size, and differences in
richness actually may be caused by differences in sample size.  To
solve this problem, we may try to rarefy species richness to the same
number of individuals.  Expected number of species in a community
rarefied from $N$ to $n$ individuals is:
\begin{equation}
\label{eq:rare}
\hat S_n = \sum_{i=1}^S (1 - p_i),\, \text{where} \quad p_i = {N-x_i
  \choose n} \Bigm /{N \choose n}
\end{equation}
where $x_i$ is the count of species $i$, and ${N \choose n}$ is the
binomial coefficient, or the number of ways we can choose $n$ from
$N$. $p_i$ give the probabilities that species $i$ does not occur in a
sample of size $n$.  This is only defined for $N-x_i > n$, but for
other cases $p_i = 0$ or the species is sure to occur in the sample.
The variance of rarefied richness is:
\begin{equation}
\label{eq:rarevar}
s^2 = p_i (1-p_i) + 2 \sum_{i=1}^S \sum_{j>i} \left[ {N- x_i - x_j
    \choose n} \Bigm / {N
    \choose n} - p_i p_j\right]
\end{equation}
Equation \ref{eq:rarevar} actually is of the same form as the variance
of sum of correlated variables:
\begin{equation}
\mathrm{var} \left(\sum x_i \right) = \sum \mathrm{var}(x_i) - 2 \mathrm{cov}(x_i, x_j)
\end{equation}

The number of stems per hectare varies in our
data set:
\begin{Schunk}
\begin{Sinput}
> quantile(rowSums(BCI))
\end{Sinput}
\begin{Soutput}
   0%   25%   50%   75%  100% 
340.0 409.0 428.0 443.5 601.0 
\end{Soutput}
\end{Schunk}
To express richness for the same number of individuals, we can use:
\begin{Schunk}
\begin{Sinput}
> Srar <- rarefy(BCI, min(rowSums(BCI)))
\end{Sinput}
\end{Schunk}
Rarefaction curves often are seen as an objective solution for
comparing species richness with different sample sizes.  However, rank
orders typically differ among different rarefaction sample sizes, and
rarefaction richness often shares the problems of Rényi diversities.

As an extreme case we may rarefy sample size to two individuals:
\begin{Schunk}
\begin{Sinput}
> S2 <- rarefy(BCI, 2)
\end{Sinput}
\end{Schunk}
This will not give equal rank order with the previous rarefaction
richness:
\begin{Schunk}
\begin{Sinput}
> all(rank(Srar) == rank(S2))
\end{Sinput}
\begin{Soutput}
[1] FALSE
\end{Soutput}
\end{Schunk}
Moreover, the rarefied richness for two individuals only is a finite
sample variant of Simpson's diversity index (or, more precisely of
$D_1 + 1$), and almost identical with sample sizes in BCI:
\begin{Schunk}
\begin{Sinput}
> range(diversity(BCI, "simp") - (S2 - 1))
\end{Sinput}
\begin{Soutput}
[1] -0.002868298 -0.001330663
\end{Soutput}
\end{Schunk}
Rarefaction is sometimes presented as ecologically meaningful
alternative to dubious diversity indices, but the differences really
seem to be small.

\section{Species abundance models}

Diversity indices may be regarded as variance measures of species
abundance distribution.  We may wish to inspect abundance
distributions more directly.  \texttt{Vegan} has functions for
Fisher's log-series and Preston's log-normal models, and in addition
several models for species abundance distribution.

\subsection{Fisher and Preston}

In Fisher's log-series, the expected number of species with $n$
individuals is:
\begin{equation}
\hat f_n = \frac{\alpha x^n}{n}
\end{equation}
where $x$ is a nuisance parameter defined by $\alpha$ and total number
of individuals $N$ in the site, $x = N/(N-\alpha)$.  Fisher's
log-series for a randomly selected plot is (Fig. \ref{fig:fisher}):
\begin{Schunk}
\begin{Sinput}
> k <- sample(nrow(BCI), 1)
> fish <- fisherfit(BCI[k, ])
> fish
\end{Sinput}
\begin{Soutput}
Fisher log series model
No. of species: 85 

      Estimate Std. Error
alpha   33.654     4.5788
\end{Soutput}
\end{Schunk}
\begin{SCfigure}
\includegraphics{diversity-vegan-015}
\caption{Fisher's log-series fitted to one randomly selected site
  (28).}
\label{fig:fisher}
\end{SCfigure}
We already saw this model as a diversity index.  Now we also obtained
estimate of standard error of $\alpha$ (these also are optionally
available in \texttt{fisher.fit}).  The standard errors are based on
the second derivatives (curvature) of the partial derivatives of
log-likelihood at the solution of $\alpha$.  The distribution of
$\alpha$ often is very non-normal and skewed, and standard errors are
of not much use.  However, \texttt{fisherfit} has a \texttt{profile}
method that can be used to inspect the validity of normal assumptions,
and will be used in calculations of confidence intervals from profile
deviance:
\begin{Schunk}
\begin{Sinput}
> confint(fish)
\end{Sinput}
\begin{Soutput}
   2.5 %   97.5 % 
25.62719 43.70514 
\end{Soutput}
\end{Schunk}

Preston's log-normal model is the main challenger to Fisher's
log-series.  Instead of plotting species by frequencies, it bins
species into frequency classes of increasing sizes.  As a result,
upper bins with high range of frequencies become more common, and
sometimes the result looks similar to Gaussian distribution truncated
at the left.

There are two alternative functions for the log-normal model:
\texttt{prestonfit} and \texttt{prestondistr}.  Function
\texttt{prestonfit} uses traditionally binning approach, and is burdened
with arbitrary choices of binning limits and treatment of ties.
Function \texttt{prestondistr} directly
maximizes truncated log-normal likelihood without binning data, and it
is the recommended alternative.  Log-normal models  usually fit poorly
to the BCI data, but here our random plot:
\begin{Schunk}
\begin{Sinput}
> prestondistr(BCI[k, ])
\end{Sinput}
\begin{Soutput}
Preston lognormal model
Method: maximized likelihood to log2 abundances 
No. of species: 85 

      mode      width         S0 
 0.9394031  1.6444133 23.4100353 

Frequencies by Octave
                0        1        2        3        4        5
Observed 31.00000 18.00000 18.00000 10.00000 4.000000 2.000000
Fitted   19.88549 23.39415 19.01393 10.67653 4.141737 1.110014
                 6
Observed 2.0000000
Fitted   0.2055267
\end{Soutput}
\end{Schunk}

\subsection{Ranked abundance distribution}

An alternative approach to species abundance distribution is to plot
logarithmic abundances in decreasing order, or against ranks of
species.  These are known, among other names, as ranked abundance
distribution curves, dominance--diversity curves and Whittaker plots.
Function \texttt{radfit} fits some of the most popular models using
maximum likelihood estimation:
\begin{align}
\hat a_r &= \frac{N}{S} \sum_{k=r}^S \frac{1}{k} &\text{brokenstick}\\
\hat a_r &= N \alpha (1-\alpha)^{r-1} & \text{preemption} \\
\hat a_r &= \exp \left[\log (\mu) + \log (\sigma) \Phi \right]
&\text{log-normal}\\
\hat a_r &= N \hat p_1 r^\gamma &\text{Zipf}\\
\hat a_r &= N c (r + \beta)^\gamma &\text{Zipf--Mandelbrot}
\end{align}
Where $\hat a_r$ is the expected abundance of species at rank $r$, $S$
is the number of species, $N$ is the number of individuals, $\Phi$ is
a standard normal function, $\hat p_1$ is the estimated proportion of
the most abundant species, and $\alpha$, $\mu$, $\sigma$, $\gamma$,
$\beta$ and $c$ are the estimated parameters in each model.

It is customary to define the models for proportions $p_r$ instead of
abundances $a_r$, but there is no reason for this, and \texttt{radfit}
is able to work with the original abundance data.  We have count data,
and the default Poisson error looks appropriate, and our example data
set gives (Fig. \ref{fig:rad}):
\begin{Schunk}
\begin{Sinput}
> rad <- radfit(BCI[k, ])
> rad
\end{Sinput}
\begin{Soutput}
RAD models, family poisson 
No. of species 85, total abundance 387

           par1      par2     par3     Deviance AIC      BIC     
Null                                   111.8736 353.0672 353.0672
Preemption  0.053337                   121.0869 364.2806 366.7232
Lognormal   0.76046   1.255             28.3779 273.5715 278.4568
Zipf        0.17283  -0.93043            8.2282 253.4219 258.3072
Mandelbrot  0.25035  -1.0368   0.58633   6.2294 253.4230 260.7510
\end{Soutput}
\end{Schunk}
\begin{SCfigure}
\includegraphics{diversity-vegan-019}
\caption{Ranked abundance distribution models for a random plot
  (no. 28).  The best model is chosen by the \textsc{aic}, and
  displayed with a thick line.}
\label{fig:rad}
\end{SCfigure}

Function \texttt{radfit} compares the models using alternatively
Akaike's or Schwartz's Bayesian information criteria.  These are based
on log-likelihood, but penalized by the number of estimated
parameters.  The penalty per parameter is $2$ in \textsc{aic}, and
$\log S$ in \textsc{bic}.  Brokenstick is regarded as a null model and
has no estimated parameters in \texttt{vegan}.  Preemption model has
one estimated parameter ($\alpha$), log-normal and Zipf models two
($\mu, \sigma$, or $\hat p_1, \gamma$, resp.), and Zipf--Mandelbrot
model has three ($c, \beta, \gamma$).

Function \texttt{radfit} also works with data frames, and fits models
for each site. It is curious that log-normal model rarely is the
choice, although it generally is regarded as the canonical model, in
particular in data sets like Barro Colorado tropical forests.

\section{Species accumulation and species pool}

Species accumulation models and species pool models study collections
of sites, and their species richness, or try to estimate the number of
unseen species.

\subsection{Species accumulation models}

Species accumulation models are similar to rarefaction: they study the
accumulation of species when the number of sites increases.  There are
several alternative methods, including accumulating sites in the order
they happen to be, and repeated accumulation in random order.  In
addition, there are three analytic models.  Rarefaction pools
individuals together, and applies rarefaction equation (\ref{eq:rare})
to these individuals.  Kindt's exact accumulator resembles rarefaction:
\begin{equation}
\label{eq:kindt}
\hat S_n = \sum_{i=1}^S (1 - p_i), \, \text{where} \quad p_i = {N- f_i
\choose n} \Bigm / {N \choose n}
\end{equation}
where $f_i$ is the frequency of species $i$.  Approximate variance
estimator is:
\begin{equation}
\label{eq:kindtvar}
s^2 = p_i (1 - p_i) + 2 \sum_{i=1}^S \sum_{j>i} \left( r_{ij}
  \sqrt{p_i(1-p_i)} \sqrt{p_j (1-p_j)}\right)
\end{equation}
where $r_{ij}$ is the correlation coefficient between species $i$ and
$j$.  Both of these are unpublished: eq. \ref{eq:kindt} was developed
by Roeland Kindt, and eq. \ref{eq:kindtvar} by Jari Oksanen. The third
analytic method was suggested by Coleman:
\begin{equation}
\label{eq:cole}
S_n = \sum_{i=1}^S (1 - p_i), \, \text{where} \quad p_i = \left(1 - \frac{1}{n}\right)^{f_i}
\end{equation}
and he suggested variance $s^2 = p_i (1-p_i)$ which ignores the
covariance component.  In addition, eq. \ref{eq:cole} does not
properly handle sampling without replacement and underestimates the
species accumulation curve.

but the recommended is Kindt's exact
method (Fig. \ref{fig:sac}):
\begin{Schunk}
\begin{Sinput}
> sac <- specaccum(BCI)
> plot(sac, ci.type = "polygon", ci.col = "yellow")
\end{Sinput}
\end{Schunk}
\begin{SCfigure}
\includegraphics{diversity-vegan-021}
\caption{Species accumulation curve for the BCI data; exact method.}
\label{fig:sac}
\end{SCfigure}

\subsection{Number of unseen species}

Species accumulation models indicate that not all potential species
are seen in any sites.  These unseen species also belong to the
species pool of the site.  Functions \texttt{specpool} and
\texttt{estimateR} implement some methods of estimating the number of
unseen species.  Function \texttt{specpool} studies a collection of
sites, and assumes how many species may be unobserved.  Function
\texttt{estimateR} works with counts of individuals, and also can be
used with a single site.  Both functions assume that the number of
unseen species is related to the number of rare species, or species
seen only once or twice.

Function \texttt{specpool} implements the following models to estimate
the pool size $S_p$:
\begin{align}
S_p &= S_o + \frac{f_1^2}{2 f_2} & \text{Chao}\\
S_p &= S_o + f_1 \frac{N-1}{N} & \text{1st order Jackknife}\\
S_p & = S_o + f_1 \frac{2N-3}{N} + f_2 \frac{(N-2)^2}{N(N-1)} &
\text{2nd order Jackknife}\\
S_p &= S_o + \sum_{i=1}^{S_o} (1-p_i)^N & \text{Bootstrap}
\end{align}
Here $S_o$ is the observed number of species, $f_1$ and $f_2$ are the
numbers of species observed once or twice, $N$ is the number of sites,
and $p_i$ are proportions of species.  The idea in jackknife seems to
be that we missed about as many species as we saw only once, and the
idea in bootstrap that if we repeat sampling (with replacement) from
the same data, we miss any many species as we missed originally.

The variance estimators are of Chao is:
\begin{equation}
s^2 = f_2 \left(\frac{G^4}{4} + G^3 + \frac{G^2}{2} \right), \,
\text{where}\quad G = \frac{f_1}{f_2}
\end{equation}
The variance of the first-order jackknife is based on the number of
``singletons'' $r$ (species occurring only once in the data) in sample
plots:
\begin{equation}
s^2 = \left(\sum_{i=1}^N r_i^2 - \frac{f_1}{N}\right) \frac{N-1}{N}
\end{equation}
Variance of the second-order jackknife is not evaluated in
\texttt{specpool} (but contributions are welcome).
For the variance of bootstrap estimator, it is practical to define a
new variable $q_i = (1-p_i)^N$ for each species:
\begin{equation}
\begin{split}
s^2 = \sum_{i=1}^{S_o} q_i (1-q_i) + 2 \sum \sum Z_p , \quad \text{where} \\
Z_p = \dots
\end{split}
\end{equation}

The extrapolated richness values for the whole BCI data are:
\begin{Schunk}
\begin{Sinput}
> specpool(BCI)
\end{Sinput}
\begin{Soutput}
    Species     Chao  Chao.SE Jack.1 Jack1.SE   Jack.2     Boot
All     225 236.6053 6.659395 245.58 5.650522 247.8722 235.6862
     Boot.SE  n
All 3.468888 50
\end{Soutput}
\end{Schunk}
If the estimation of pool size really works, we should get the same
values of estimated richness if we take a random subset of a half of
the plots:
\begin{Schunk}
\begin{Sinput}
> s <- sample(nrow(BCI), 25)
> specpool(BCI[s, ])
\end{Sinput}
\begin{Soutput}
    Species     Chao  Chao.SE Jack.1 Jack1.SE   Jack.2     Boot
All     212 242.0312 14.23972 241.76 8.528165 256.1733 225.7051
     Boot.SE  n
All 4.684375 25
\end{Soutput}
\end{Schunk}
These typically are even lower than the observed richness
(225 species) at the whole data set.

\subsection{Pool size from a single site}

The \texttt{specpool} function needs a collection of sites, but there
are some methods that estimate the number of unseen species for each
single site.  These functions need counts of individuals, and species
seen only once or twice, or other rare species, take the place of
species with low frequencies.  Function \texttt{estimateR} implements
two of these methods:
\begin{Schunk}
\begin{Sinput}
> estimateR(BCI[k, ])
\end{Sinput}
\begin{Soutput}
                 28
S.obs     85.000000
S.chao1  109.473684
se.chao1  12.578970
S.ACE    116.301606
se.ACE     5.509697
\end{Soutput}
\end{Schunk}
Chao's method is similar as above, but uses another, ``unbiased''
equation. \textsc{ace} is based on rare species also:
\begin{equation}
\begin{split}
S_p &= S_\mathrm{abund} + \frac{S_\mathrm{rare}}{C_\mathrm{ACE}} +
\frac{a_1}{C_\mathrm{ACE}} \gamma^2 \quad \text{where}\\
C_\mathrm{ACE} &= 1 - \frac{a_1}{N_\mathrm{rare}}\\
\gamma^2 &= \frac{S_\mathrm{rare}}{C_\mathrm{ACE}} \sum_{i=1}^{10} i
(i-1) a_1 \frac{N_\mathrm{rare} - 1}{N_\mathrm{rare}}
\end{split}
\end{equation}
Now $a_1$ takes the place of $f_1$ above, and means the number of
species with only one individual.
Here $S_\mathrm{abund}$ and $S_\mathrm{rare}$ are the numbers of
species of abundant and rare species, with an arbitrary upper limit of
10 individuals for a rare species, and $N_\mathrm{rare}$ is the total
number of individuals in rare species.

The pool size
is estimated separately for each site, but if input is a data frame,
each site will be analysed.

If log-normal abundance model is appropriate, it can be used to
estimate the pool size.  Log-normal model has a finite number of
species which can be found integrating the log-normal:
\begin{equation}
S_p = S_\mu \sigma \sqrt{2 \pi}
\end{equation}
where $S_\mu$ is the modal height or the expected number of species at
maximum (at $\mu$), and $\sigma$ is the width.  Function
\texttt{veiledspec} estimates this integral from a model fitted either
with \texttt{prestondistr} or \texttt{prestonfit}, and fits the latter
if raw site data are given.  Log-normal model fits badly, and
\texttt{prestonfit} is particularly poor.  Therefore the following
explicitly uses \texttt{prestondistr}, although this also may fail:
\begin{Schunk}
\begin{Sinput}
> veiledspec(prestondistr(BCI[k, ]))
\end{Sinput}
\begin{Soutput}
Extrapolated     Observed       Veiled 
    96.49459     85.00000     11.49459 
\end{Soutput}
\begin{Sinput}
> veiledspec(BCI[k, ])
\end{Sinput}
\begin{Soutput}
Extrapolated     Observed       Veiled 
    406.4778      85.0000     321.4778 
\end{Soutput}
\end{Schunk}

\subsection{Probability of pool membership}

Beals smoothing was originally suggested as tool of regularizing data
for ordination.  It regularizes data too strongly for that purpose,
but it has been suggested as a method of estimating which of the
missing species could occur in a site, or which sites are suitable for
a species.  The probability for each species at each site is assessed
from other species occurring on the site.

Function \texttt{beals} implement Beals smoothing:
\begin{Schunk}
\begin{Sinput}
> smo <- beals(BCI)
\end{Sinput}
\end{Schunk}
We may see how the estimated probability of occurrence and observed
numbers of stems relate in one of the more familiar species
(Fig. \ref{fig:beals}):
\begin{Schunk}
\begin{Sinput}
> j <- which(colnames(BCI) == "Ceiba.pentandra")
> plot(smo[, j], BCI[, j], main = "Ceiba pentandra", xlab = "Probability of occurrence", 
+     ylab = "Occurrence")
\end{Sinput}
\end{Schunk}
\begin{SCfigure}
\includegraphics{diversity-vegan-028}
\caption{Beals smoothing for \emph{Ceiba pentandra}.}
\label{fig:beals}
\end{SCfigure}

\end{document}
